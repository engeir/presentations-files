\subsection{Concluding remarks}

\begin{frame}{Concluding remarks}

  \begin{itemize}[<+->]
    \item It is possible to obtain a response function from a
      non-parametric method
    \item Historic temperature is reproduced from forcing data
    \item Reproduces shape of \(4\times\ce{CO2}\) experiment \(\rightarrow\) infer
      climate sensitivity
    % \item Power spectral density of residuals fit nicely to the PSD of a
    %   control run
  \end{itemize}

  \note<1>{
  \begin{itemize}
    \item We saw that it is possible to obtain a response function from a non-parametric method
  \end{itemize}
  }
  \note<2>{
  \begin{itemize}
    \item Using this response function, temperature over the last two millennia
      was reconstructed from forcing data
  \end{itemize}
  }
  \note<3>{
  \begin{itemize}
    \item Finally, using a constant forcing, temperature from a
      \(4\times\ce{CO2}\) experiment was estimated
    \item The estimates had similar shape to the temperature from simulation,
      suggesting the possibility of inferring climate sensitivity from the
      response function
  \end{itemize}
  }

\end{frame}

\subsection{Future work}

\begin{frame}{Future work}

  \begin{itemize}[<+->]
    \item Analysis of the shape of the response function, for example using
      exponentials or power laws
    \item More exact study of the \(4\times\ce{CO2}\) experiment
    \item Effect of clustering of volcanoes
    % \item Estimation of the shape through the deconvolution algorithm using our
    %   own climate simulations
  \end{itemize}
  \vspace{-2mm}
  \begin{figure}
    \centering
    % \includegraphics<1>[width=\linewidth]{../figures/response_func_noresm_all_dark.pdf}
    \includegraphics<1>[width=\linewidth]{../figures/response_func_noresm1_choose_dark.pdf}
    % \includegraphics<2>[width=\linewidth]{../figures/estimate_historic/temp_abrupt1_dark.pdf}%
    \includegraphics<2>[width=\linewidth]{../figures/estimate_historic/temp_abrupt1_alone_dark.pdf}%
    \includegraphics<3->[width=\linewidth]{../figures/raw_historical_beam.pdf}%
    % \includegraphics<3>[width=\linewidth]{../figures/estimate_historic/best_fit_raw_temps_dark.pdf}%
  \end{figure}

  {\pgfsetfillopacity{0.35}
    \begin{tikzpicture}[visible on=<4>,overlay]
    \draw[fill = lightgray] ([xshift=8.46cm,yshift=4.89cm]current page text area.east) rectangle (7.47,1.85);
    \end{tikzpicture}
  }
  \pgfsetfillopacity{1}

  \note<1>{
  \begin{itemize}
    \item Even though the shape is similar in the three versions, it is not
      easy to get an idea of the underlying shape
    \item Settling on one version and investigating what shape it has is one
      future plan
  \end{itemize}
  }
  \note<2>{
  \begin{itemize}
    \item We also saw the constant forcing experiment was well reproduced, but
      this is also short and do not have very high temporal resolution
    \item A more detailed study is therefore sought after
  \end{itemize}
  }
  \note<3>{
  \begin{itemize}
    \item Before the little ice age there are partucularly many strong
      volcanoes, suggesting they might play a role
    \item Studying how such clustering of volcanoes affect climate on longer
      time scales is therefore also included here as future work
  \end{itemize}
  }

\end{frame}
